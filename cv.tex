% Copyright (C) 2019-2021 by Cheng XU <copyright@xuc.me>
% All Rights Reserved.

\documentclass{mycv}

\name[Zuokun OUYANG]{Zuokun OUYANG \large{Ph.D.}}
\phone{+86 186-0022-6561\\}
\email{zuokun.ouyang@outlook.com}
\homepage{https://alainouyang.github.io}
%\github{xu-cheng}
\wechat{oyzk2012}
\linkedin{zuokun-ouyang}
% \address{School of Computing Science, Simon Fraser University \\ 8888 University Drive, Burnaby, BC V5A 1S6, Canada}

\usepackage{soul}

\newcommand{\myname}[1]{\bf \ul{Z. Ouyang}}

\begin{document}

\maketitle

% \section{Research \\ Interests}

% As a recent Ph.D. graduate, my research interests focus on the intersection of econometrics and machine learning for time series forecasting. My current research includes the following aspects:

% \begin{itemize}
%   \itemsep 0em
%   \item Time series analysis \& forecasting.
%   \item Econometric and machine learning integration.
%   \item Sequential \& temporal learning.
% \end{itemize}

\begin{summary}
  Recently graduated with a Ph.D., I specialize in the fusion of econometrics and machine learning, particularly in time series forecasting. My research area encompasses time series analysis, sequential learning, and econometrics, in which I bring a robust understanding of both theoretical concepts and practical applications. I am also a team player with a proven track record of collaborating with cross-functional teams to achieve business goals.
\end{summary}

\vspace{-1em}

\section{Education}

\subsection{\large \scshape University of Orl\'eans}[Orl\'eans, France]

\begin{positions}
  \entry{Ph.D.\footnotemark[1], Computer Science and Signal Processing}{Oct. 2019~--~Sept. 2023}
\end{positions}

\begin{itemize}
  \itemsep 0em
  \item Dissertation: \textit{Time Series Forecasting: From Econometrics to Deep Learning}
  \item Supervisors: Prof.~Philippe Ravier, Assoc.~Prof.~Meryem Jabloun
\end{itemize}

\vspace{-\parskip}

\subsection{\large \scshape University of Orl\'eans}[Orl\'eans, France]

\begin{positions}
  \entry{Dipl\^ome d'Ing\'enieur\footnotemark[2], Computer Engineering, {\it Polytech Orl\'eans}}{Sept. 2015~--~Sept. 2018}
  \entry{M.Sc., Computer Science}{Sept. 2017~--~Sept. 2018}
\end{positions}

\begin{itemize}
  \itemsep 0em
  \item Dissertation: \textit{A Fundamental Study on Deep Learning based Time Series Forecasting}
  \item Supervisors: Prof.~Christel Vrain, Prof.~Marcilio C. P. de Souto, Assoc.~Prof.~Sylvie Treuillet
\end{itemize}

\vspace{-\parskip}

\subsection{\large \scshape Beijing Institute of Technology}[Beijing, China]

\begin{positions}
  \entry{B.Eng., Electrical \& Electronics Engineering}{Sept. 2012~--~June 2016}
\end{positions}

\begin{itemize}
  \itemsep 0em
  \item Dissertation: \textit{A Microphone Array-based System for Sound Source Localization}
  \item Supervisors: Assoc.~Prof.~Shiyong Li, Assoc.~Prof.~Rodolphe Weber
\end{itemize}

\vspace{-1em}

\section{Professional \\ Experience}

\subsection{\large \scshape University of Orl\'eans}[Orl\'eans, France]

\begin{positions}
  \entry{Lecturer, {\it Teaching for Engineering Program at Polytech Orl\'eans}}{Jan. 2023~--~Sept. 2023}
\end{positions}

\begin{itemize}
  \itemsep 0em
  \item Signals and Linear Systems \textit{(EPL3CI13)}.
  \item Introduction to Signal Processing \textit{(EPL4CI04)}.
  \item Acquisition Systems \textit{(EPL2IA01)}.
  \item Arduino \& Embedded Systems \textit{(EPL2CI03)}.
\end{itemize}

\vspace{-\parskip}

\subsection{\large \scshape ATTILA Gestion}[Lyon \& Montargis, France]

\begin{positions}
  \entry{Data Scientist}{Oct. 2019~--~Dec. 2022}
\end{positions}

\begin{itemize}
  \itemsep 0em
  \item Worked with cross-functional teams to develop a forecasting tool to support business decisions.
  \item Developed an evaluation framework to assess the performance of forecasting models on different sampling frequencies, seasonalities, and stationarity, under different forecasting strategies.
  \item Proposed STLformer, a Transformer-based time series forecasting model. It uses Spearman's $\rho$-based attention and STL decomposition, combined with the ARCH effect test, to achieve forecasting in $\mathcal{O}(N \log{N})$. It outperforms SOTA methods on multiple datasets, especially on nonlinearly dependent series (e.g., financial series). Deployed in the internal forecasting tool.
\end{itemize}

\vspace{-\parskip}

\subsection{\large \scshape ATTILA Gestion}[Montargis, France]

\begin{positions}
  \entry{Data Scientist Intern}{Apr. 2018~--~Sept. 2018}
\end{positions}

\begin{itemize}
  \itemsep 0em
  \item Assessed various internal metrics with time series tools to evaluate franchisees' performance.
  \item Systematically reviewed commonly used methods in time series analysis, e.g., ARIMA, ETS, Theta, and decomposition methods and assessed them on the company's data.
\end{itemize}

\vspace{-\parskip}

\subsection{\large \scshape eContent Store S\`arl}[Luxembourg]

\begin{positions}
  \entry{Software Development Intern}{June 2017~--~Aug. 2017}
\end{positions}

\begin{itemize}
  \itemsep 0em
  \item Acted as one of the core developers of the Android development team.
  \item Implemented key enhancements and upgrades for AR functionalities, encompassing improved technique selection, natural feature training pipeline, and numerous bug fixes.
  \item Developed a WebGL tool for natural features training to improve rendering performance.
  \item Wrote design and related interface documentation, and user manuals for the WebGL tool.
\end{itemize}

\footnotetext[1]{Industrial Ph.D. program contracted with ATTILA Gestion.}

\section{Skills}

\begin{description}
  \itemsep 0em
  \item[Programming:] Python, R, C\#, Java, C/C++, Swift, MATLAB, SQL
  \item[Frameworks \& Tools:] PyTorch, scikit-learn, Unity3D, OpenCV, PowerBI, Linux, Git
  \item[Skills \& Expertise:] Deep Learning, Machine Learning, Time Series Analysis, Econometrics, Causal Inference, Signal Processing, Optimization Theory, Non-linear Regression
  \item[Languages:] English (proficient), French (proficient), Mandarin (native)
\end{description}

\section{Selected \\ Publications}

\publications[keyword={selected}]{CV collections.bib}
% \publications[keyword={selected}]{publications.bib}
\vspace{-0.5em}
\section{Selected \\ Projects}

\subsection{iOS Application \textit{RestauRank}}[Mar. 2018~--~Apr. 2018]

\begin{itemize}
  \itemsep 0em
  \item Developed a map App to locate top-rated local restaurants and determine the fastest route.
  \item Google Maps SDK and Google Geolocation API for map visualization, navigation, and reviews.
\end{itemize}

\vspace{-0.5em}

\subsection{Archaeological Ceramic Decorations Segmentation}[Jan. 2018~--~Mar. 2018]

\begin{itemize}
  \itemsep 0em
    \item Built 2D FCNs to segment decorated regions on ancient ceramic fragments using depth maps.
    \item Clustered segmented areas and preprocessed depth maps into distinct categories.
    \item Benchmarked the clustering results against other algorithms, including $K$-means and DBSCAN.
\end{itemize}

\vspace{-0.5em}

\subsection{Plant ECGs Classification with CNN and SVM}[Oct. 2017~--~Dec. 2017]

\begin{itemize}
  \item Sampled 400 plant ECG signals using BitScope and plant ECG sensor.
  \item Extracted four features from the ECGs and classified with 1D-CNN and SVM.
  \item Achieved 87\% accuracy with 1D-CNN and 98\% accuracy with SVM.
\end{itemize}

\vspace{-0.5em}

\subsection{A Microphone Array-based System for Sound Source Localization}[Mar. 2016~--~May 2016]

\begin{itemize}
  \itemsep 0em
  \item Developed a microphone array system with Python for sound source localization.
  \item Used Raspberry Pi, Arduino UNO, a stepper motor, and an eight-microphone array.
  \item Implemented DOA-TDOA \& GCC algorithms for sound source localization.
\end{itemize}

% \section{Talks}

% \begin{itemize}
%   \itemsep 0em
%   \item Use Time Series Prediction Methods to Forecast Customers Number, \emph{1st Collaborative Workshop on Artificial Intelligence Applications for Small Medium Enterprises}, Orl\'eans, France, June~2018.
% \end{itemize}

\vspace{-0.5em}

\section{Awards}

\begin{itemize}
  \itemsep 0em
  \item Erasmus+ Consortium Polytech, Polytech Orléans \printdate{2017}
  \item College Student Academic Scholarship, Beijing Institute of Technology \printdate{2012~--~2015}
  % \item Provincial 2nd Prize, Beijing University Students' Calligraphy and Painting Exhibition \printdate{2014}
  \item National 3rd Prize, Chinese Exhibition of Calligraphy and Painting for Undergraduates \printdate{2013}
  \item National 3rd Prize, The 25th Chinese Chemistry Olympiad \printdate{2011}
  \item Provincial 1st Prize, The 28th Chinese Physics Olympiad \printdate{2011}
  \item Provincial 1st Prize, The 20th China High School Biology Olympiad \printdate{2011}
\end{itemize}

\vspace{-0.5em}

\section{Other \\ Experience}
\begin{itemize}
  \itemsep 0em
  \item  Volunteer, Chinese New Year Festivity, \emph{Orl\'eans and Yangzhou Government} \printdate{Feb.~2017}
  \item Vice President, \emph{Association of Calligraphy of Beijing Institute of Technology} \printdate{2013~--~2015}
\end{itemize}

\vspace{-0.5em}

\section{Hobbies}
Basketball, Reading, Chinese Calligraphy, Singing, Fitness, and Cooking.

\footnotetext[2]{The ``Diplôme d'Ingénieur'' is a highly accredited elite diploma in France. Equivalent to M.Eng. Only the top 10\% of the students in the French Baccalaureate can apply for this education program in engineering schools. Students delve deep into engineering and science courses and receive management, economics, and social sciences education, ensuring they emerge as well-rounded professionals.}
\end{document}
