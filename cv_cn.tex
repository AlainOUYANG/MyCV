% Copyright (C) 2019-2021 by Cheng XU <copyright@xuc.me>
% All Rights Reserved.

\documentclass{mycv}

\name[欧阳佐坤]{欧阳佐坤~{\normalsize 求职意向: 算法研究员. 随时入职}}
\phone{+86 186-0022-6561\\}
\email{zuokun.ouyang@outlook.com}
\homepage{https://alainouyang.github.io}
\wechat{oyzk2012}
\linkedin{zuokun-ouyang}
% \address{\textbf{求职意向: 数据科学家. 十月初可入职}}

\usepackage{soul}

\newcommand{\myname}[1]{\bf \ul{Z. Ouyang}}

\begin{document}

\maketitle

\begin{summary}
  2023 年 9 月取得博士学位, 专长于计量经济学和深度学习的交叉领域, 特别在时间序列预测方面有深入的研究. 研究涵盖时间序列分析、信号处理及序列学习等领域. 致力于将理论与业务结合, 创造实际价值. 在团队合作方面有着丰富的经验, 能与不同领域的团队成员良好合作, 共同达成目标. 对自己的领域充满热情, 期待在未来带来更多的价值和创新.
\end{summary}

% \vspace{-1em}

\section{教育背景}

\subsection{\large 奥尔良大学}[奥尔良, 法国]

\begin{positions}
  \entry{{\sffamily 哲学博士}, 计算机科学与技术}{10/2019~--~09/2023}
\end{positions}

\begin{itemize}
  \itemsep 0.4em
  \item 学位论文: \textit{Time Series Forecasting: From Econometrics to Deep Learning}
  \item 指导教师: Philippe Ravier 教授, Meryem Jabloun 副教授
  % \item 项目资金: Association Nationale de la Recherche et de la Technologie CIFRE $N^{\circ}$ 2019/0551
  % \item 与 ATTILA Gestion 公司合作, 企业博士项目
\end{itemize}

\vspace{-\parskip}

\subsection{\large 奥尔良大学}[奥尔良, 法国]

\begin{positions}
  \entry{{\sffamily 工程师文凭}\footnotemark[1], 计算机工程, 综合理工学院}{09/2015~--~09/2018}
  \entry{{\sffamily 理学硕士}, 计算机科学, 计算机学院}{09/2017~--~09/2018}
\end{positions}

\begin{itemize}
  \itemsep 0.4em
  \item 学位论文: \textit{A Fundamental Study on Deep Learning based Time Series Forecasting}
  \item 指导教师: Christel Vrain 教授, Marcilio C. P. de Souto 教授, Sylvie Treuillet 副教授
\end{itemize}

\vspace{-\parskip}

\subsection{\large 北京理工大学}[北京, 中国]

\begin{positions}
  \entry{{\sffamily 工学学士}, 电子信息工程}{09/2012~--~06/2016}
\end{positions}

\begin{itemize}
  \itemsep 0.4em
  \item 学位论文: \textit{A Microphone Array-based System for Sound Source Localization}
  \item 指导教师: Shiyong Li 副教授, Rodolphe Weber 副教授
\end{itemize}

\vspace{-1em}

\section{工作经历}

\subsection{\large 奥尔良大学}[奥尔良, 法国]

\begin{positions}
  \entry{助教, 综合理工学院}{01/2023~--~08/2023}
\end{positions}

为本科生和硕士研究生讲授以下课程:

\begin{itemize}
  \itemsep 0.4em
  \item 信号处理: 信号处理导论、信号与线性系统、采样系统与信号处理.
  \item 嵌入式系统: Arduino Programming.
\end{itemize}

\vspace{-\parskip}

\subsection{\large \scshape ATTILA Gestion}[里昂, 法国]

\begin{positions}
  \entry{数据科学家}{04/2018~--~12/2022}
\end{positions}

\begin{itemize}
  \itemsep 0.4em
  \item ATTILA 是法国一家连锁加盟企业, 专注于屋顶的维修和保养. 其客户包括了法国多家知名企业, 如 Orange, Carrefour, Total 等. 公司内部有着大量的时序数据, 如客户数量、收入、服务类型等.
  \item 针对不同规模、领域的客户群体, 公司需要对其进行细分, 以分配不同的资源和服务. 同时需要为客户数量、订单量、收入等关键指标开发内部预测工具, 以支持业务决策和各地加盟商的运营.
  \item 使用多种聚类方法进行了客户细分, 并使用计量经济学和深度学习方法, 对客户数量、订单量、收入等关键指标进行预测. 为内部的预测工具提供了多种预测模型, 并在内部数据上进行了评估.
  \item 研究了分解集成策略对于不同预测方法在性能上的影响, 实现了多种预测策略下的深度学习模型的评估框架, 用于评估其在不同特征(\emph{长度、粒度、季节性、平稳性})的数据集上的性能.
  \item 开发了一套自动化预测工具, 针对不同特征的时序数据, 自动选择最优的预测策略和模型: 针对不超过 100 个数据点的短序列, 选择传统方法如 ARIMA, Theta, 并采用集成学习策略; 针对长度在 100 至 500 之间的中等序列, 选择 Hybrid 方法如 ES-RNN, Prophet, 并采用分解集成策略; 针对长度超过 500 的长序列, 选择深度学习方法如 Transformer.
  \item 基于 Flask, Plotly 和 sktime 等库, 开发了一款 Web APP, 以 Docker 形式交付, 用于预测客户数量、订单量、收入等关键指标. 用户无须懂得时序相关的技术, 只需要上传符合格式的 Excel 文件, 此工具即可完成预测分析, 并将预测结果以图表的形式展示给用户, 支持其业务决策.
\end{itemize}

% \vspace{-\parskip}

% \subsection{\large \scshape ATTILA Gestion}[蒙塔尔纪, 法国]

% \begin{positions}
%   \entry{数据科学家, 实习}{04/2018~--~09/2018}
% \end{positions}

% \begin{itemize}
%   \itemsep 0em
%   \item 通过使用时序分析工具对连锁加盟分部的各种内部指标进行分析, 以评估其发展状况.
%   \item 对时序分析中常用的方法, 如 ARIMA, ETS, 时序分解, 以及常用的计量经济学和机器学习模型, 如 State Space, Theta, SVR 等, 进行了系统的文献综述, 并在公司内部的数据上进行了评估.
% \end{itemize}

\vspace{-\parskip}

\subsection{\large \scshape eContent Store S\`arl}[卢森堡]

\begin{positions}
  \entry{软件开发, 实习}{06/2017~--~08/2017}
\end{positions}

\begin{itemize}
  \itemsep 0.4em
  \item 作为公司 AR 产品 Android 开发团队的核心开发者之一, 开发了 AR 产品的一系列关键功能和升级措施, 同时提出并应用了更好的技术选型和更高效的自然特征训练流程.
  \item 为了优化渲染性能, 专门设计并开发了一个优化自然特征训练流程的 WebGL 工具.
  \item 撰写了设计文档和相关的接口文档, 以及 WebGL 工具的用户手册.
\end{itemize}

\section{项目经历}

\subsection{STLformer: 基于 STL 分解与 Rank Correlation 的时序预测模型}[03/2023]

\begin{itemize}
  \item 提出了 STLformer, 一个基于 Transformer 的时间序列预测模型.
  \item 使用 STL 对序列进行滚动分解, 并分别以编、解码器对序列中的季节性和长期趋势进行建模.
  \item 基于 ARCH 效应检验, 判断出序列是否具有异方差性, 并在此基础上提出了基于 Spearman 相关系数的自注意力机制(Rank Correlation), 对序列中的非线性依赖进行建模.
  \item STLformer 实现了在 $\mathcal{O}(N \log{N})$ 复杂度下的时间序列预测, 在多个数据集上均取得了 SOTA 的预测性能, 尤其在具有非线性依赖的序列(如金融时序)上有明显提升 ($\thicksim 21\%$).
\end{itemize}

\vspace{-\parskip}

\subsection{基于深度学习的时间序列预测策略研究}[07/2022]

\begin{itemize}
  \item 评估了多种预测策略的优劣和应用场景, 如一步预测、滚动预测、直接预测、MIMO 以及 MISMO.
  \item 在多个时序预测任务、不同的预测策略上, 对比评估了多个不同的深度学习模型的效果表现.
  \item 针对时序数据的长度、粒度、季节性、平稳性等特征, 讨论了不同模型和策略的优劣, 并针对不同的应用场景提出了相应的模型与策略的选择建议.
\end{itemize}

\vspace{-\parskip}

\subsection{分解集成策略对时序预测算法的影响研究}[07/2021]

\begin{itemize}
  \item 评估了几种常用的时序分解算法的优劣和应用场景, 如 Classical、STL 以及 Prophet.
  \item 使用分解集成策略, 将这些分解算法与传统的计量经济学方法和机器学习方法相结合.
  \item 在 M-Competition 数据集上对比评估了多个不同模型的效果表现.
  \item 分解集成策略能够显著提升\textit{传统方法}的预测性能, 但对\textit{机器学习方法}的影响取决于数据分布.
\end{itemize}

% \subsection{iOS 地图应用 \textit{RestauRank}}[03/2018~--~04/2018]

% \begin{itemize}
%   \itemsep 0em
%   \item 一款 iOS 地图应用, 能够定位当地受欢迎和高评价的餐馆, 并为用户推荐最快到达的路线.
%   \item 利用 Google Maps SDK 和 Google Geolocation API 进行地图可视化、导航以及评论获取功能.
% \end{itemize}

\vspace{-\parskip}

\subsection{基于深度学习的古代陶瓷碎片图像分割}[02/2018]

\begin{itemize}
  \itemsep 0em
  \item 构建基于二维全卷积网络的分割模型, 对古代陶瓷碎片的扫描图上的装饰区域进行分割.
  \item 将此二维全卷积网络的分割结果与其他算法(如 $K$-均值和 DBSCAN)进行基准测试对比.
\end{itemize}

\vspace{-\parskip}

\subsection{基于卷积神经网络和支持向量机的植物 ECG 信号分类}[11/2017]

\begin{itemize}
  \item 使用 BitScope 和专用的植物 ECG 传感器采集了 400 组植物的 ECG 信号数据.
  \item 分别采用 1D-CNN 和 SVM 进行信号分类, 其中 SVM 使用从 ECG 信号中提取的四个核心特征.
  \item 1D-CNN 模型达到 87\%的分类准确率,基于 SVM 的分类器则达到 98\% 的准确率.
\end{itemize}

% \subsection{基于麦克风阵列的声源定位系统}[03/2016~--~05/2016]

% \begin{itemize}
%   \itemsep 0em
%   \item 基于 Python 和 C++ 语言开发了一个用于声源定位的麦克风阵列系统.
%   \item 系统基于树莓派、Arduino UNO、步进电机以及一个八通道的麦克风阵列.
%   \item 使用 NumPy 实现了 DOA-TDOA/GCC 算法, 使用 Arduino UNO 控制步进电机指向声源方向.
% \end{itemize}

% \section{Talks}

% \begin{itemize}
%   \itemsep 0em
%   % \item STL Decomposition of Time Series Can Benefit Forecasting Done by Statistical Methods but Not by Machine Learning Ones, \emph{7th International conference on Time Series and Forecasting}, Gran Canaria, Spain, July~2021.
%   \item Use Time Series Prediction Methods to Forecast Customers Number, \emph{1st Collaborative Workshop on Artificial Intelligence Applications for Small Medium Enterprises}, Orl\'eans, France, June~2018.
% \end{itemize}

\vspace{-\parskip}

\section{学术论文}

\publications[keyword={selected}]{CV collections.bib}
% \publications[keyword={selected}]{publications.bib}

\vspace{-\parskip}

\section{专业技能}

\begin{description}
  \itemsep 0em
  \item[{\it 编程语言}:] Python, R, MATLAB, C\#, Java, C/C++, SQL, \LaTeX
  \item[{\it 工具与框架}:] PyTorch, scikit-learn, Unity3D, OpenCV, PowerBI, Linux, Git
  \item[{\it 技能专长}:] 深度学习, 机器学习, 时间序列分析, 计量经济学, 因果推断, 信号处理, 非线性优化
  \item[{\it 外语}:] 英语(听说读写流利), 法语(听说读写流利)
\end{description}

\vspace{-\parskip}

\section{获奖情况}

\begin{itemize}
  \itemsep 0em
  \item Erasmus+ 奖学金, \textit{Erasmus+ Consortium Polytech} \printdate{2017}
  \item 优秀学生奖学金, \textit{北京理工大学} \printdate{2012~--~2015}
  \item 国家级三等奖, \textit{中国高校书画摄影联展} \printdate{2013}
\end{itemize}

\vspace{-\parskip}

\section{其他经历}
\begin{itemize}
  \itemsep 0em
  \item 志愿者, 中国新年庆祝活动, \emph{奥尔良与扬州市政府} \printdate{02/2017}
  \item 副社长, \emph{北京理工大学书画社} \printdate{2013~--~2015}
\end{itemize}

% \section{兴趣爱好}
% 篮球, 阅读, 书法, 唱歌, 健身, 烹饪.

\footnotetext[1]{工程师文凭在法国是一个被高度认证的精英学历, 相当于工程硕士学位. 只有法国高考前 10\% 的学生才有资格申请工程师学院的教育项目. 除了大量工程和科学课程外, 学生还会接受管理、经济和社会科学的教育, 确保成为全面的专业人士.}
\end{document}
