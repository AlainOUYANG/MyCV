% Copyright (C) 2019-2021 by Cheng XU <copyright@xuc.me>
% All Rights Reserved.

\documentclass{mycv}

\name[欧阳佐坤]{欧阳佐坤~\normalsize{博士. 计算机科学与技术}}
\phone{+86 186-0022-6561\\}
\email{zuokun.ouyang@outlook.com}
\homepage{https://alainouyang.github.io}
\wechat{oyzk2012}
\linkedin{zuokun-ouyang}

\usepackage{soul}

\newcommand{\myname}[1]{\bf \ul{Z. Ouyang}}

\linespread{1.2}

\begin{document}

\maketitle

% \begin{summary}
%   2023 年 9 月取得博士学位, 专长于计量经济学和深度学习的交叉领域, 特别在时间序列预测方面有深入的研究. 研究涵盖时间序列分析、信号处理及序列学习等领域. 致力于将理论与业务结合, 创造实际价值. 在团队合作方面有着丰富的经验, 能与不同领域的团队成员良好合作, 共同达成目标. 对自己的领域充满热情, 期待在未来带来更多的价值和创新.
% \end{summary}

% \vspace{-1em}

\section{教育背景}

\subsection{\large 奥尔良大学}[奥尔良, 法国]

\vspace{0.3em}

\begin{positions}
  \entry{{\sffamily 工学博士}, 计算机科学与技术}{2019 年 10 月~--~2023 年 9 月}
\end{positions}

\begin{itemize}
  \item 学位论文: \textit{Time Series Forecasting: From Econometrics to Deep Learning}
  % \item 指导教师: Philippe Ravier 教授, Meryem Jabloun 副教授
  % \item 项目资金: Association Nationale de la Recherche et de la Technologie CIFRE $N^{\circ}$ 2019/0551
  % \item 与 ATTILA Gestion 公司合作, 企业博士项目
\end{itemize}

\vspace{-\parskip}

\subsection{\large 奥尔良大学}[奥尔良, 法国]

\vspace{0.3em}

\begin{positions}
  \entry{{\sffamily 工程师文凭}\footnotemark[1], 计算机工程, 综合理工学院}{2015 年 9 月~--~2018 年 9 月}
  \entry{{\sffamily 理学硕士}, 计算机科学, 计算机学院}{2017 年 9 月~--~2018 年 9 月}
\end{positions}

\vspace{-\parskip}

\subsection{\large 北京理工大学}[北京, 中国]

\vspace{0.3em}

\begin{positions}
  \entry{{\sffamily 工学学士}, 电子信息工程}{2012 年 9 月~--~2016 年 6 月}
\end{positions}

\vspace{-1em}

\section{工作经历}

\subsection{\large 美团}[北京, 中国]

\vspace{0.3em}

\begin{positions}
  \entry{核心本地商业, 搜索与推荐技术部, 营销增长算法工程师(L7)}{2023 年 12 月~--~至今}
\end{positions}

\begin{itemize}
  \item 带领 3 人团队, 深度参与美团神会员券包售卖方向的算法定价策略的优化升级, 负责下沉市场整体业务, 同时负责成本调控系统的迭代升级.
\end{itemize}

\vspace{-0.5em}

\subsection{\large 奥尔良大学}[奥尔良, 法国]

\vspace{0.3em}

\begin{positions}
  \entry{综合理工学院, 助教}{2023 年 1 月~--~2023 年 8 月}
\end{positions}

\begin{itemize}
  \item 为本科生和硕士研究生讲授以下课程: 信号处理导论、信号与线性系统、采样系统与信号处理.
\end{itemize}

\vspace{-0.5em}

\subsection{\large \scshape ATTILA Gestion}[里昂, 法国]

\vspace{0.3em}

\begin{positions}
  \entry{数据科学家}{2018 年 4 月~--~2022 年 12 月}
\end{positions}

\begin{itemize}
  \item ATTILA 是法国一家连锁加盟企业, 专注于屋顶的维修和保养. 其客户包括法国多家大型企业, 如 Orange, Carrefour, Total 等. 负责对公司不同规模、领域的客户群体进行细分, 以分配不同的资源和服务. 同时对客户数量、订单量、收入等关键指标进行预测, 以支持业务决策和运营.
\end{itemize}

\vspace{-0.5em}

% \subsection{\large \scshape ATTILA Gestion}[蒙塔尔纪, 法国]

% \begin{positions}
%   \entry{数据科学家, 实习}{04/2018~--~09/2018}
% \end{positions}

% \begin{itemize}
%   \itemsep 0em
%   \item 通过使用时序分析工具对连锁加盟分部的各种内部指标进行分析, 以评估其发展状况.
%   \item 对时序分析中常用的方法, 如 ARIMA, ETS, 时序分解, 以及常用的计量经济学和机器学习模型, 如 State Space, Theta, SVR 等, 进行了系统的文献综述, 并在公司内部的数据上进行了评估.
% \end{itemize}

% \vspace{-\parskip}

% \subsection{\large \scshape eContent Store S\`arl}[卢森堡]

% \begin{positions}
%   \entry{软件开发, 实习}{06/2017~--~08/2017}
% \end{positions}

% \begin{itemize}
%   \itemsep 0.4em
%   \item 作为公司 AR 产品 Android 开发团队的核心开发者之一, 开发了 AR 产品的一系列关键功能和升级措施, 同时提出并应用了更好的技术选型和更高效的自然特征训练流程.
%   \item 为了优化渲染性能, 专门设计并开发了一个优化自然特征训练流程的 WebGL 工具.
%   \item 撰写了设计文档和相关的接口文档, 以及 WebGL 工具的用户手册.
% \end{itemize}

\section{项目经历}

\subsection{神券包售卖算法定价策略提效}[2024 年 5 月~--~至今]

\begin{itemize}
  \itemsep 0.4em
  \item 神券包是美团神会员的重要提频工具, 旨在利用用户的损失厌恶心理来提升增长目标. 策略通过差异化发放抵扣券, 间接降低券包价格, 促进高价值用户的转化, 从而提升增长目标.
  \item 家店融合前期, 运营侧采用人工操盘方式, 对不同订单 \& 客单价组合的人群分层进行差异化投放, 而策略采用「ML + OR」的两阶段框架对不同投放动作(Treatment)进行 Uplift 建模. 主要难点在于: 1. 决策空间动作多; 2. 预算水平变化大; 3. 实付建模难度高.
  \item v1.0 策略受限于 RCT 数据量, 选择以订单为目标进行建模; 同时针对离在线成本差异大的情况, 设计了高补贴力度的 Treatment 映射组合, 扩大成本决策空间; 线上同时对 1. 用户分层粒度拉格朗日分配、2. 用户粒度的拉格朗日分配和 3. 泛化能力强的、基于启发式分配的 0-1 Treatment 策略进行 A/B 实验. 最终方案 2 获得推全, 拉动 CLC 大盘订单 +0.96\%, 实付 +0.42\%.
  \item v2.0 策略在 v1.0 的基础上, 主要优化实付 GTV 指标. 难点在于, 实付 GTV 的波动大, 长尾效应明显, 信噪比低. 因此将实付 GTV 目标拆解为 $\text{GTV} = \text{订单} \times \text{客单价}$, 并分别对订单和客单价进行建模. 同时对客单价进行凹函数变换, 在概率密度的主体部分提升增量信号,在极端值部分降低噪声,从而提升建模目标的信噪比. 此外, 将两阶段求解框架替换为端到端的 DFL 方案. 此组合方案获得推全, 拉动 CLC 大盘订单 +0.71\%, 实付 +0.32\%.
  \item v2.2 策略采用 MMoE 模型, 使用多目标建模的方式, 对不同业务线进行差异化建模. 此方案获得推全, 拉动 CLC 大盘订单 +0.35\%, 实付 +0.26\%, 策略效果持续提升.
\end{itemize}

% \vspace{-\parskip}

% \subsection{基于 MPC 的成本调控系统}[2025 年 4 月~--~至今]

% \begin{itemize}
%   \itemsep 0.4em
%   \item 成本调控系统是券定价系统的重要组成部分,旨在通过调控 line 值来达到调整整组优惠力度的目的.
%   \item 结果:实现了基于 MPC 的成本调控系统的上线,并取得了显著的提效效果.
% \end{itemize}

\vspace{-\parskip}

\subsection{自动化时序预测工具}[2023 年 7 月]

\begin{itemize}
  \itemsep 0.4em
  \item 使用聚类方法对 ATTILA 的客户进行细分, 并使用计量经济学和深度学习方法, 对客户数量、订单量、收入等指标进行预测和评估.
  \item 研究了分解集成策略对于不同预测方法在性能上的影响, 实现了多种预测策略下的深度学习模型的评估框架, 用于评估其在不同特征(\emph{长度、粒度、季节性、平稳性})的数据集上的性能.
  \item 开发了一套自动化预测工具, 针对不同特征/长度的时序数据, 自动选择最优的预测策略(\emph{集成学习、分解集成策略})和模型(\emph{ARIMA、Theta、ES-RNN、Prophet、Transformer}).
  \item 基于 Flask, Plotly 和 sktime 等库, 将上述方法封装为 Web 应用, 使用者只需上传格式匹配的 Excel 文件, 即可完成预测分析, 结果以不同维度的图表展示, 支持业务决策和运营.
\end{itemize}

\vspace{-\parskip}

\subsection{STLformer: 基于 STL 分解与 Rank Correlation 的时序预测模型}[2023 年 3 月]

\begin{itemize}
  \itemsep 0.4em
  \item 提出了 STLformer, 一个基于 Transformer 的时间序列预测模型.
  \item 使用 STL 对序列进行滚动分解, 并分别以编、解码器对序列中的季节性和长期趋势进行建模.
  \item 基于 ARCH 效应检验, 判断出序列是否具有异方差性, 并在此基础上提出了基于 Spearman 相关系数的自注意力机制(Rank Correlation), 对序列中的非线性依赖进行建模.
  \item STLformer 实现了在 $\mathcal{O}(N \log{N})$ 复杂度下的时间序列预测, 在多个数据集上均取得了 SOTA 的预测性能, 尤其在具有非线性依赖的序列(如金融时序)上有明显提升 (+21\%).
\end{itemize}

\vspace{-\parskip}

\subsection{分解集成策略对时序预测算法的影响研究}[2021 年 7 月]

\begin{itemize}
  \itemsep 0.4em
  \item 评估了几种常用的时序分解算法的优劣和应用场景, 如 Classical、STL 以及 Prophet.
  \item 使用分解集成策略, 将这些分解算法与传统的计量经济学方法和机器学习方法相结合.
  \item 在 M-Competition 数据集上对比评估了多个不同模型的效果表现.
  \item 分解集成策略能够显著提升\textit{传统方法}的预测性能, 但对\textit{机器学习方法}的影响则取决于数据分布.
\end{itemize}

% \subsection{iOS 地图应用 \textit{RestauRank}}[03/2018~--~04/2018]

% \begin{itemize}
%   \itemsep 0em
%   \item 一款 iOS 地图应用, 能够定位当地受欢迎和高评价的餐馆, 并为用户推荐最快到达的路线.
%   \item 利用 Google Maps SDK 和 Google Geolocation API 进行地图可视化、导航以及评论获取功能.
% \end{itemize}

\vspace{-\parskip}

% \subsection{基于深度学习的古代陶瓷碎片图像分割}[02/2018]

% \begin{itemize}
%   \itemsep 0em
%   \item 构建基于二维全卷积网络的分割模型, 对古代陶瓷碎片的扫描图上的装饰区域进行分割.
%   \item 将此二维全卷积网络的分割结果与其他算法(如 $K$-均值和 DBSCAN)进行基准测试对比.
% \end{itemize}

% \vspace{-\parskip}

% \subsection{基于卷积神经网络和支持向量机的植物 ECG 信号分类}[11/2017]

% \begin{itemize}
%   \item 使用 BitScope 和专用的植物 ECG 传感器采集了 400 组植物的 ECG 信号数据.
%   \item 分别采用 1D-CNN 和 SVM 进行信号分类, 其中 SVM 使用从 ECG 信号中提取的四个核心特征.
%   \item 1D-CNN 模型达到 87\%的分类准确率, 基于 SVM 的分类器则达到 98\% 的准确率.
% \end{itemize}

% \subsection{基于麦克风阵列的声源定位系统}[03/2016~--~05/2016]

% \begin{itemize}
%   \itemsep 0em
%   \item 基于 Python 和 C++ 语言开发了一个用于声源定位的麦克风阵列系统.
%   \item 系统基于树莓派、Arduino UNO、步进电机以及一个八通道的麦克风阵列.
%   \item 使用 NumPy 实现了 DOA-TDOA/GCC 算法, 使用 Arduino UNO 控制步进电机指向声源方向.
% \end{itemize}

% \section{Talks}

% \begin{itemize}
%   \itemsep 0em
%   % \item STL Decomposition of Time Series Can Benefit Forecasting Done by Statistical Methods but Not by Machine Learning Ones, \emph{7th International conference on Time Series and Forecasting}, Gran Canaria, Spain, July~2021.
%   \item Use Time Series Prediction Methods to Forecast Customers Number, \emph{1st Collaborative Workshop on Artificial Intelligence Applications for Small Medium Enterprises}, Orl\'eans, France, June~2018.
% \end{itemize}

% \vspace{-\parskip}

\section{学术论文}

\publications[keyword={selected}]{CV collections.bib}
% \publications[keyword={selected}]{publications.bib}

\vspace{-\parskip}

\section{技术栈}

\begin{description}
  \item[\textit{编程语言}:] Python, Java, Spark/Hive, SQL, \LaTeX
  \item[\textit{工具与框架}:] TensorFlow, PyTorch, scikit-learn, Unity3D, OpenCV, PowerBI, Linux, Git
  \item[\textit{技能专长}:] 机器学习, 时间序列分析, Uplift 建模, 因果推断, 信号处理, 非线性优化
  \item[\textit{外语}:] 英语(听说读写流利), 法语(听说读写流利)
\end{description}

\vspace{-\parskip}

\section{获奖情况}

\begin{itemize}
  \item Erasmus+ 奖学金, \textit{Erasmus+ Consortium Polytech} \printdate{2017}
  \item 连续三年优秀学生奖学金, \textit{北京理工大学} \printdate{2012~--~2015}
  \item 行草书法作品, 国家级三等奖, \textit{中国高校书画摄影联展} \printdate{2013}
\end{itemize}

\vspace{-\parskip}

\section{其他经历}
\begin{itemize}
  \item 志愿者, 中国新年庆祝活动, \emph{奥尔良与扬州市政府} \printdate{2017 年 2 月}
  \item 副社长, \emph{北京理工大学书画社} \printdate{2013~--~2015}
\end{itemize}

\section{兴趣爱好}
篮球, 阅读, 书法, 唱歌, 健身, 烹饪.

\footnotetext[1]{工程师文凭在法国是一个被高度认可的精英学历, 相当于工程硕士学位. 只有法国高考前 10\% 的学生才有资格申请工程师学院的教育项目. 除了大量工程和科学课程外, 学生还会接受管理、经济和社会科学的教育, 确保成为全面的专业人士.}
\end{document}
